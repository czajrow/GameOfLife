\documentclass[a4paper,12pt,oneside]{article}

\usepackage[utf8]{inputenc}
\usepackage[pdftex]{graphicx}
\usepackage{polski}
\usepackage{amsfonts}
\usepackage{verbatim}
\usepackage{indentfirst}
\usepackage{listings}
\usepackage[polish]{babel}
\usepackage[T1]{fontenc}
\usepackage{fancyhdr}
\usepackage{lastpage}
\pagestyle{fancy}
\renewcommand{\headrulewidth}{0pt}
\rhead{}
\lhead{}
\cfoot{str. \thepage/\pageref{LastPage}}

\begin{document}
\makeatletter
\addtocontents{toc}{\protect\thispagestyle{empty}}
\newcommand{\linia}{\rule{\linewidth}{0.4mm}}

\renewcommand{\maketitle}{\begin{titlepage}

    \vspace*{1cm}

    \begin{center}\small

    Politechnika Warszawska\\

    Wydział Elektryczny\\

    \end{center}

    \vspace*{1cm}

    \noindent\linia

    \begin{center}

      \LARGE \textsc{\@title}

         \end{center}

      \noindent\linia

    \vspace{1cm}

    \begin{flushright}

    \begin{minipage}{5cm}

    \textit{\small Autorzy:}\\

    \normalsize \textsc{\@author} \par

    \end{minipage}
         \end{flushright}
 \begin{center}
    \vspace{5cm}

     {\small Praca wykonana w~ramach przedmiotu:
     Języki i~metody programowania 2}

    \vspace*{\stretch{6}}

    \@date

    \end{center}

  \end{titlepage}%

}

\makeatother

\title{Specyfikacja funkcjonalna}

\title{SPECYFIKACJA FUNKCJONALNA\\projekt: automat komórkowy}

\author{Anna Głowińska\newline numer indeksu: 291070\newline anna.glowinska98@gmail.com \newline \newline
Adam Czajka\newline numer indeksu: 291063 \newline czajka.adam147@gmail.com}

\maketitle
\tableofcontents
\thispagestyle{fancy}
\newpage
\section{Opis ogólny}

\subsection{Nazwa programu}

\textit{Life} - automat komórkowy symulujący „Grę w~życie” Johna Conwaya zbudowany w~języku~C.

\subsection{Podstawowe pojęcia}

\noindent   \textbf{Komórka} - najmniejsza jednostka automatu komórkowego; może znajdować się w~stanie martwym (przypisujemy jej kolor biały~=~0) lub w stanie żywym (kolor czarny~=~1).
\\\\
\textbf{Automat komórkowy} - uporządkowany zbiór komórek, z~których każda znajduje się w~jednym z~kilku dozwolonych stanów (np. martwy i żywy). Komórki przylegają do siebie tworząc siatki.
\\\\
\textbf{Siatka} - dwuwymiarowy kawałek płaszczyzny o~konkretnych wymiarach, podzielony liniami tworzącymi macierz, składającą się z~pojedynczych komórek.
\\\\
\textbf{Generacja} - Stan wszystkich komórek w~danej chwili, ale też czynność zmiany ich stanów.
\\\\
\textbf{Sąsiedztwo} - graniczenie komórki z~innymi komórkami, warunkujące nowy stan tej komórki, wyróżniamy: \newline
\begin{itemize}
\item sąsiedztwo Moore'a: 8 przylegających komórek (znajdujących się: na południu, na południowym-zachodzie, na zachodzie, na północnym-zachodzie, na północy, na północnym-wschodzie, na wschodzie i~na południowym-wschodzie),
\item sąsiedztwo von Neumanna: 4 przylegające komórki (na południu, zachodzie, północy i~wschodzie).
\end{itemize}

\subsection{Poruszany problem}
Zadaniem automatu komórkowego \textit{Life} jest symulacja „Gry w~życie” wymyślonej i~opracowanej przez Johna Hortona Conwaya. Gra toczy się na dwuwymiarowej płaszczyźnie podzielonej na kwadratowe komórki. W~zależności od przyjętych zasad każda komórka ma czterech sąsiadów przylegających do niej bokami (sąsiedztwo von Neumanna) lub ośmiu sąsiadów (sąsiedztwo Moore'a), czyli komórki przylegające do niej bokami i~rogami. Każda komórka może znajdować się w~jednym z~dwóch stanów: może być albo żywa (włączona), albo martwa (wyłączona).
\par Stany komórek zmieniają się przy każdej generacji. Dana generacja jest używana do obliczenia następnej generacji. Po obliczeniu wszystkie komórki zmieniają swój stan dokładnie w~tym samym momencie. Stan komórki po generacji zależy tylko od liczby jej żywych sąsiadów.
\par Zestaw zasad przy tworzeniu nowej generacji (przejścia do następnego stanu) jest następujący:


\begin{enumerate}
\item Martwa komórka, która ma dokładnie 3~żywych sąsiadów, staje się żywa w~następnym opisie (rodzi się).
\item Żywa komórka z~2~albo 3~żywymi sąsiadami pozostaje nadal żywa; przy innej liczbie sąsiadów umiera (z~samotności albo zatłoczenia).
\end{enumerate}





\section{Opis funkcjonalności}


\subsection{Korzystanie z programu}
Program \textit{Life} jest kompilowany oraz uruchamiany za pomocą programu \verb+make+, który czyta co ma zrobić z~pliku o~nazwie \verb+Makefile+, znajdującego się w katalogu \verb+life+.


\subsection{Uruchomienie programu}
Program \verb+make+ wywoływany jest bez poleceń z~linii komend. Wszelkie parametry podane przez użytkownika przy wywoływaniu zostaną przez program zignorowane.

\subsection{Możliwości programu}

Zadania jakie wykonuje program:
\begin{itemize}
\item  wczytywanie do programu początkowej konfiguracji generacji z~pliku w~formacie TXT,\hspace{6in}
\item  wczytanie danych wejściowych z wbudowanej w program biblioteki,\hspace{6in}
\item  przeprowadzenie zadanej liczby generacji,\hspace{6in}
\item  zapisywanie bieżącej generacji do pliku TXT, który może zostać potem wczytany,\hspace{6in}
\item  zapisywanie bieżącej generacji do pliku PBM.
\end{itemize}


\section{Format danych i~struktury plików}


\subsection{Struktura katalogów}

Szkielet projektu został przedstawiony na poniższym schemacie: \newline

\par life
\par +--- dane
\par \ |\ \ \ \   +--- dane.txt
\par \ |\ \ \ \   +--- biblioteka
\par \ |\ \ \ \   +--- wyniki
\par +--- src
\par \ |\ \ \ \   +--- testy
\par +--- bin
\par +--- Makefile
\newline

gdzie: \par \verb+life+ - katalog główny,
\par \verb+dane+ - katalog zawierający dane testowe (dane.txt) oraz katalog,
\par \verb+biblioteka+ - zawierający ciekawe przykłady danych wejściowych,
\par \verb+wyniki+ - katalog w którym zostają zapisane bieżące generacje w~formacie TXT oraz obrazy przedstawiające stan wybranych generacji w~postaci plików PBM,
\par \verb+src+ - katalog zawierający pliki z~treścią kodu,
\par \verb+testy+ - katalog zawierający pliki z kodem poszczególnych testów,
\par \verb+bin+ - katalog zawierający plik wykonywalny,
\par \verb+Makefile+ - plik zawierający treść reguły programu \verb+make+.


\subsection{Przechowywanie danych}
\textbf{Na~dysku:}
Pliki z~danymi tekstowymi oraz obrazy przedstawiające stan wybranych generacji są zapisywane w~katalogu \verb+life/dane/wyniki+. Zostaną nazwane zgodnie z odpowiadającymi im numerami generacji.
\newline
\par \textbf{W~programie:}
Zostanie utworzona struktura przechowująca dane o~wielkości płaszczyzny zadanej przez użytkownika oraz wektor przechowujący kolejne liczby (0 i 1), odpowiadające kolejnym martwym i~żywym komórkom.


\subsection{Dane wejściowe}

Program będzie operował na pliku wejściowym o~formacie TXT, gdzie dane przedstawione są w postaci:\newline
\par \verb+x y+
\par \verb+1 0 1 1 …+
\par \verb+0 0 0 1 …+
\par \verb+0 1 0 0 …+
\par \verb+… … … +\newline
\par gdzie:
\begin{itemize}
\item \verb+x+, \verb+y+ to odpowiednio podane szerokość oraz długość dwuwymiarowej płaszczyzny na której odbędzie się symulacja „Gry w~życie”,
\item \verb+0+~–~odpowiada komórce martwej, \verb+1+~-~żywa.
\end{itemize}


\subsection{Dane wyjściowe}

Program dzięki swojej pracy generuje :\begin{enumerate}
\item Obrazy przedstawiających stan wybranych generacji w~postaci plików PBM.
\item Bieżącą generację w~pliku TXT.
\end{enumerate}
\par Te pliki zostaną zapisane w~katalogu: \verb+life/dane/wyniki+.

\section{Scenariusz działania programu}
\subsection{Scenariusz ogólny}
Przebieg działania programu:
\begin{enumerate}
\item Kompilacja (o~ile jest konieczna - zgonie z zasadami działania programu \verb+make+) i~uruchomienie.
\item Pobranie od użytkownika informacji o~rodzaju danych początkowych (własnych albo z biblioteki), sąsiedztwie, którego zamierza użyć dla kolejnych generacji oraz o~liczbie generacji, jaką chce przeprowadzić.
\item Przebieg generacji.
\item Pobranie od użytkownika informacji o~zapisie bieżącej generacji w pliku TXT lub w~formie graficznej (plik PBM) oraz jej ewentualny zapis.
\item Powtórzenie punktów 3. oraz 4. tyle razy, ile zadał użytkownik.
\item Zakończenie działania programu.
\end{enumerate}
\subsection{Scenariusz szczegółowy}
Szczegółowy przebieg działania programu:
\begin{enumerate}
\item Uruchomienie programu przebiega poprzez wywołanie programu \verb+make+ - bez podawania argumentów przez użytkownika. W przypadku ich podania zostają one całkowicie zignorowane.
\item Program oferuje możliwość wyboru między sąsiedztwem Moore'a oraz von~Neumanna poprzez wpisanie odpowiednio liczby \verb+1+ lub \verb+2+. Następnie pyta o liczbę generacji do wykonania oraz o rodzaj danych wejściowych. W przypadku wybrania danych z biblioteki program wypisze na ekran pełną listę nazw dostępnych przykładowych  danych, które znajdują się w katalogu \verb+life/dane/biblioteka+, z których należy wybrać jeden wpisując odpowiednią liczbę.
\item Każdorazowo po przebiegu generacji i wyświatleniu jej na ekran program umożliwia użytkownikowi zapis bieżącej generacji do pliku tekstowego lub graficznego. Następuje to poprzez wpisaniu odpowiedniego znaku. Wpisanie \verb+s+ zapisuje plik PBM, \verb+t+ zapisuje plik TXT, \verb+q+ natychmiast kończy działanie programu, a znaki od \verb+1+ do \verb+9+ przechodzą o odpowiednią liczbę generacji.
\item Po wykonaniu odpowiedniej liczby generacji program samoczynnie kończy działanie.
\end{enumerate}

\section{Testowanie}

\subsection{Ogólny przebieg testowania}

Do weryfikacji poprawności kodu zastosowane zostaną testy cząstkowe sprawdzające prawidłowość działania odpowiednich funkcji programu \textit{Life}.
\par Pliki zawierające kod poszczególnych testów będą się znajdować w katalogu \verb+life/src/testy+.

\end{document}
